\documentclass[12pt, oneside]{article} 
% \documentclass{article} 
\usepackage{amsmath, amsthm, amssymb, calrsfs, wasysym, verbatim, bbm, 
                color, graphics, geometry}
\usepackage{fancyvrb} % for "\Verb" macro
\usepackage[toc,page]{appendix}
\usepackage[pdftex]{graphicx}
\usepackage{float}
\usepackage{hyperref}
\usepackage{mathtools}
\hypersetup{
    colorlinks=true,
    linkcolor=blue,
    filecolor=magenta,      
    urlcolor=cyan,
}
 
\urlstyle{same}

\geometry{tmargin=1.25in, bmargin=1.25in, lmargin=1.00in, rmargin = 1.00in}  

\newcommand{\R}{\mathbb{R}}
\newcommand{\C}{\mathbb{C}}
\newcommand{\Z}{\mathbb{Z}}
\newcommand{\N}{\mathbb{N}}
\newcommand{\Q}{\mathbb{Q}}
\newcommand{\Cdot}{\boldsymbol{\cdot}}

\usepackage{listings}
\usepackage{color} %red, green, blue, yellow, cyan, magenta, black, white
\definecolor{mygreen}{RGB}{28,172,0} % color values Red, Green, Blue
\definecolor{mylilas}{RGB}{170,55,241}

\title{CIS580 PSet 3}
\author{Sheil Sarda}
\date{02.24.2021}

\begin{document}
\lstset{language=Matlab,%
    %basicstyle=\color{red},
    breaklines=true,%
    morekeywords={matlab2tikz},
    keywordstyle=\color{blue},%
    morekeywords=[2]{1}, keywordstyle=[2]{\color{black}},
    identifierstyle=\color{black},%
    stringstyle=\color{mylilas},
    commentstyle=\color{mygreen},%
    showstringspaces=false,%without this there will be a 
    % symbol in the places where there is a space
    numbers=left,%
    numberstyle={\tiny \color{black}},% size of the numbers
    numbersep=9pt, % this defines how far the numbers are from the text
    emph=[1]{for,end,break},emphstyle=[1]\color{red}, %some words to emphasise
    %emph=[2]{word1,word2}, emphstyle=[2]{style},    
}

\begin{titlepage}
    \begin{flushleft}
        \vspace*{1cm}
        \Huge
        \textbf{CIS580 Problem Set 6\\ }
        \vspace*{0.5cm}
        \normalsize
        Sheil Sarda \verb|<sheils@seas.upenn.edu>| \\
        CIS580 Spring 2021
        \tableofcontents
    \end{flushleft}
\end{titlepage}

\section{Mapping 2D points}

\subsection{Use solution for orthogonal Procrustes problem to obtain rotation 
matrix}

Centroid of first set of four points: $(1.25, 1.50)$ and of the 
second set of four points: $(1.80, 1.11)$.

\begin{align*}
    A    &= 
    \begin{bmatrix} -1.25 & -0.25 & -0.25 & 1.75\\ \\ 
                    -0.50 & -1.50 & -0.50 & 2.50
    \end{bmatrix} \\ \\
    B    &= 
    \begin{bmatrix} -0.65 & -1.52 & -0.53 & 2.70\\ \\ 
                    1.18 &  0.07 &  0.19 & -1.43
    \end{bmatrix} 
\end{align*}

Taking the SVD of $A \times B^{T}$

\begin{align*}
    [U, S, V^T] &= SVD(A \times B^{T}) \\ \\
    \implies R &= V \times
    \begin{bmatrix} 1 & 0 \\ \\ 
        0 & det(V \times U^T) 
    \end{bmatrix}  \times U^T \\ \\
    &= \begin{bmatrix} 0.1234 & 0.9924 \\ \\ 
        -0.9924 & 0.1234
    \end{bmatrix}  
\end{align*}

\subsection{Find solution that solves directly for the rotation angle $\theta$
and translation $[T_x, T_Y]$}

\begin{align*}
    \sum_{}{} &({x_i}' - x_i \cdot cos(\theta) + y_i \cdot sin(\theta) -t_x)^2 \\
            +&({y_i}' - x_i \cdot sin(\theta) - y_i \cdot cos(\theta) -t_y)^2
\end{align*}

Taking partial derivatives with respect to $t_x$:
\begin{align*}
    \frac{\partial }{\partial t_x} &=  -2 \cdot (-t_x - x_i \cdot cos(\theta) 
    + y_i \cdot sin(\theta) + x') = 0 \\ \\
    \implies t_x &= -\bar{x_i} \cdot cos(\theta) + 
                    \bar{y_i} \cdot sin(\theta) + \bar{x'}
\end{align*}

Taking partial derivatives with respect to $t_y$:
\begin{align*}
    \frac{\partial }{\partial t_y} &=  -2 \cdot (-t_y - x_i \cdot sin(\theta) 
    + y_i \cdot cos(\theta) + y') = 0 \\ \\
    \implies t_y &= -\bar{x_i} \cdot sin(\theta) -
                    \bar{y_i} \cdot cos(\theta) + \bar{y'}
\end{align*}

Next steps: 
\begin{enumerate}
    \item Combine the equations for $t_x$ and $t_y$ using the original equation
    \item Take the derivative with respect to $\theta$ and set it equal to 0
    \item Solve for $t_x$ and $t_y$ by substituting $\theta$
\end{enumerate}


\section{Phone held vertically}

\subsection{Write projection equations}

Given:
\begin{align*}
    \lambda \cdot
    \begin{bmatrix} x \\ \\ y \\ \\ 1 \end{bmatrix}  
    &= 
    \begin{bmatrix} cos(\theta) & 0 & sin(\theta) \\ \\ 
                    0 & 1 & 0\\ \\ 
                    -sin(\theta) & 0 & cos(\theta)
    \end{bmatrix}  \cdot 
    \begin{bmatrix} X \\ \\ Y \\ \\ Z \end{bmatrix}  +
    \begin{bmatrix} T_X \\ \\ T_Y \\ \\ T_Z \end{bmatrix}  
\end{align*}

Using the above, we can generate a system of 4 equations with 4 unknowns as 
follows:
\begin{align*}
    x_2 &= \frac{a \cdot cos(\theta) + T_X}{-a \cdot sin(\theta) + T_Z} \\
    y_2 &= \frac{0 + T_Y}{-a \cdot sin(\theta) + T_Z} \\
    x_1 &= \frac{T_X}{T_Z}  \\
    y_1 &= \frac{T_Y}{T_Z} 
\end{align*}

\subsection{Solve equations for yaw angle $\theta$ and translations 
$[T_x, T_y, T_z]$}

Solving the above system of equations:
\begin{align*}
T_X &= x_1 \cdot {T_Z}  \\
x_2 &= \frac{a \cdot cos(\theta) + x_1 \cdot {T_Z}}{-a \cdot sin(\theta) + T_Z} \\
T_Z \cdot (x_2 - x_1) &= a \cdot cos(\theta) + a \cdot sin(\theta) \cdot x_2 \\
T_Z &= \frac{a \cdot cos(\theta) + a \cdot sin(\theta) \cdot x_2}{x_2 - x_1} \\
\end{align*}

Another equation in terms of $T_Z$ and $\theta$ can be obtained as follows:
\begin{align*}
T_Y &= y_1 \cdot {T_Z}  \\
y_2 &= \frac{0 + T_Y}{-a \cdot sin(\theta) + T_Z} \\
T_Z \cdot (y_2 - y_1) &= a \cdot sin(\theta) \cdot y_2 \\
T_Z &= \frac{a \cdot sin(\theta) \cdot y_2}{y_2 - y_1} \\
\end{align*}

Solving for $\theta$:
\begin{align*}
\frac{a \cdot sin(\theta) \cdot y_2}{y_2 - y_1}
&= \frac{a \cdot cos(\theta) + a \cdot sin(\theta) \cdot x_2}{x_2 - x_1} \\
cos(\theta) &= -x_2 \cdot y_1 + x_1 \cdot y_2 \\
sin(\theta) &= y_1 - y_2  \\ \\
\implies \theta &= arctan2 \left( 
    \frac{y_1 - y_2}{-x_2 \cdot y_1 + x_1 \cdot y_2}
    \right) \\
\end{align*}

Using this to solve for the other unknowns:
\begin{align*}
    T_Z &= \frac{a \cdot sin(\theta) \cdot y_2}{y_2 - y_1} \\
    T_X &= x_1 \cdot {T_Z}  
        &= \frac{x_1 \cdot a \cdot sin(\theta) \cdot y_2}{y_2 - y_1} \\
    T_Y &= y_1 \cdot {T_Z}  
        &= \frac{y_1 \cdot a \cdot sin(\theta) \cdot y_2}{y_2 - y_1}\\
\end{align*}

\subsection{Conditions on camera position to obtain unique or finite number 
of solutions}
$y_1 \neq y_2 \implies$ the length of the rod cannot be zero.

\clearpage
\section{Decompose $H$ into rotation $R$ and translation $T$}

Given:

\begin{align*}
    & g_1 =  
    \begin{bmatrix} 0 \\ \\ \frac{\sqrt{3}}{2} \\ \\ \frac{1}{2} \end{bmatrix} 
    & g_2 =  
    \begin{bmatrix} 0 \\ \\ \frac{1}{2} \\ \\ \frac{\sqrt{3}}{2} \end{bmatrix} 
\end{align*}

\begin{align*}
    H =  
    \begin{bmatrix} \frac{-4}{8} & 0 & 1 \\ \\ 
                    \frac{6}{8} & \frac{\sqrt{3} - 12}{8} 
                    & \frac{5 -4\sqrt{3}}{8}    \\ \\ 
                    \frac{-2 \sqrt{3}}{8} & 
                    \frac{7 + 4\sqrt{3}}{8} & \frac{\sqrt{3} + 4}{8} 
                \end{bmatrix} 
\end{align*}

Also, from the provided picture we can obtain the rotation matrices $R_1$ and 
$R_2$:

\begin{align*}
    & R_1 =  
    \begin{bmatrix} 1 & 0 & 0 \\ \\ 
                    0 & \frac{\sqrt{3}}{2} & -\frac{1}{2}\\ \\ 
                    0 & \frac{1}{2} & \frac{\sqrt{3}}{2} \end{bmatrix} 
    & R_2 =  
    \begin{bmatrix} 1 & 0 & 0 \\ \\ 
                    0 & \frac{1}{2} & -\frac{\sqrt{3}}{2}\\ \\ 
                    0 & \frac{\sqrt{3}}{2} & \frac{1}{2} \end{bmatrix} 
\end{align*}


\begin{align*}
    H'  &= {R_2}^{-1} \times H \times R_1 \\ \\
        &=
        \begin{bmatrix} \frac{-1}{2}  & \frac{1}{2} & \frac{\sqrt{3}}{2} \\ 
            0   & 1 & 0 \\ 
            -\frac{\sqrt{3}}{2}  & 2 & \frac{-1}{2} \end{bmatrix} 
\end{align*}

We know $H'$ is a composition of a rotation about the $Y$ axis, followed by a 
translation. Thus, we can also represent it as:

\begin{align*}
    H'  &=
        \begin{bmatrix} cos(\theta)  & 0 & sin(\theta) \\ \\
            0  & 1 & 0 \\  \\
            -sin(\theta)   & 0 & cos(\theta) 
        \end{bmatrix} +
        \begin{bmatrix} 0  & \frac{T'_X}{2} & 0 \\ \\
            0  & \frac{T'_Y}{2} & 0 \\ \\
            0  & \frac{T'_Z}{2} & 0 
        \end{bmatrix} 
\end{align*}

Equating the two version of $H'$, we find:
\begin{align*}
    \begin{bmatrix} \frac{T'_X}{2} & \frac{T'_Y}{2} & \frac{T'_Z}{2} 
    \end{bmatrix}  &= 
    \begin{bmatrix} 1 & 0 & 4 
    \end{bmatrix} \\  \\
    \theta &= \frac{2 \cdot \pi }{3}
\end{align*}

We also know that the rotation matrix to go from Camera 1 to Camera 2 in Camera
1 coordinates $R_Y$ is a rotation about the $Y$ axis by $\theta$ radians. Thus:

\begin{align*}
    & R_Y =  
    \begin{bmatrix} \frac{-1}{2}  &  0 & \frac{\sqrt{3}}{2}\\ \\
         0   & 1 & 0 \\ \\
         \frac{-\sqrt{3}}{2}  &  0 & \frac{-1}{2}\end{bmatrix} 
\end{align*}

Combining the above information, we can create $ \prescript{1}{}{\mathbf{R}_2}:$

\begin{align*}
    & \prescript{1}{}{\mathbf{R}_2} 
    =  R_1 \times R_Y \times {R_2}^{-1}
    = \begin{bmatrix} -0.500  & -0.750 & 0.433 \\ \\ 
                    0.433   & 0.217 & 0.875 \\ \\ 
                    -0.750   & 0.625 & 0.217 \end{bmatrix} 
\end{align*}

Now we use the above information to extract the translation vector from $H$.
\begin{align*}
    T   &= R_1 \times T' \\ \\
        &=  \begin{bmatrix} 1 & 0 & 0 \\ \\ 
            0 & \frac{\sqrt{3}}{2} & -\frac{1}{2}\\ \\ 
            0 & \frac{1}{2} & \frac{\sqrt{3}}{2} \end{bmatrix}  \times
            \begin{bmatrix} 1  \\ \\ 
                0 \\ \\ 
                4 \end{bmatrix}  \\ \\
        &=  \begin{bmatrix} 1.000  \\ \\ 
                -2.000 \\ \\ 
                3.464 \end{bmatrix}  \\ \\
\end{align*}

Thus, $H$ can be decomposed into  $ \prescript{1}{}{\mathbf{R}_2}$ and $T$.

\end{document}

