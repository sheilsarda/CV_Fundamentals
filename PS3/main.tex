\documentclass[12pt, oneside]{article} 
% \documentclass{article} 
\usepackage{amsmath, amsthm, amssymb, calrsfs, wasysym, verbatim, bbm, color, graphics, geometry}
\usepackage{fancyvrb} % for "\Verb" macro
\usepackage[toc,page]{appendix}
\usepackage[pdftex]{graphicx}
\usepackage{float}
\usepackage{hyperref}
\hypersetup{
    colorlinks=true,
    linkcolor=blue,
    filecolor=magenta,      
    urlcolor=cyan,
}
 
\urlstyle{same}

\geometry{tmargin=1.25in, bmargin=1.25in, lmargin=1.00in, rmargin = 1.00in}  

\newcommand{\R}{\mathbb{R}}
\newcommand{\C}{\mathbb{C}}
\newcommand{\Z}{\mathbb{Z}}
\newcommand{\N}{\mathbb{N}}
\newcommand{\Q}{\mathbb{Q}}
\newcommand{\Cdot}{\boldsymbol{\cdot}}

\usepackage{listings}
\usepackage{color} %red, green, blue, yellow, cyan, magenta, black, white
\definecolor{mygreen}{RGB}{28,172,0} % color values Red, Green, Blue
\definecolor{mylilas}{RGB}{170,55,241}

\title{CIS580 PSet 2}
\author{Sheil Sarda}
\date{02.14.2021}

\begin{document}
\lstset{language=Matlab,%
    %basicstyle=\color{red},
    breaklines=true,%
    morekeywords={matlab2tikz},
    keywordstyle=\color{blue},%
    morekeywords=[2]{1}, keywordstyle=[2]{\color{black}},
    identifierstyle=\color{black},%
    stringstyle=\color{mylilas},
    commentstyle=\color{mygreen},%
    showstringspaces=false,%without this there will be a symbol in the places where there is a space
    numbers=left,%
    numberstyle={\tiny \color{black}},% size of the numbers
    numbersep=9pt, % this defines how far the numbers are from the text
    emph=[1]{for,end,break},emphstyle=[1]\color{red}, %some words to emphasise
    %emph=[2]{word1,word2}, emphstyle=[2]{style},    
}

\begin{titlepage}
    \begin{flushleft}
        \vspace*{1cm}
        \Huge
        \textbf{CIS580 Problem Set 2\\ }
        \vspace*{0.5cm}
        \normalsize
        Sheil Sarda \verb|<sheils@seas.upenn.edu>| \\
        CIS580 Spring 2021
        \tableofcontents
    \end{flushleft}
\end{titlepage}

\section{Line that passes through the points}

\subsection{$[0, a, 0], [0, 0, a]$}
  \begin{align*}
    l &= \begin{bmatrix}
           0 \\
           a \\
           0
         \end{bmatrix} \times 
          \begin{bmatrix}
                    0 \\
                    0 \\
                    a
                  \end{bmatrix}  
= 
 \begin{bmatrix}
                    a^2 - 0 \\
                    0 - 0 \\
                    0 - 0
                  \end{bmatrix}                   
= 
 \begin{bmatrix}
                    a^2  \\
                    0  \\
                    0 
                  \end{bmatrix}                                                   
  \end{align*}

\subsection{$[a, a, 1], [a, a, 2]$}
  \begin{align*}
    l &= \begin{bmatrix}
           a \\
           a \\
           1
         \end{bmatrix} \times 
          \begin{bmatrix}
                    a \\
                    a \\
                    2
                  \end{bmatrix}  
= 
 \begin{bmatrix}
                    2a - a \\
                    a - 2a \\
                    a^2 - a^2
                  \end{bmatrix}                   
= 
 \begin{bmatrix}
                    a  \\
                    -a  \\
                    0 
                  \end{bmatrix}                                                   
  \end{align*}

\subsection{$[a, b, 0], [c, d, 0]$}
  \begin{align*}
    l &= \begin{bmatrix}
           a \\
           b \\
           0
         \end{bmatrix} \times 
          \begin{bmatrix}
                    c \\
                    d \\
                    0
                  \end{bmatrix}  
= 
 \begin{bmatrix}
                    0 -0 \\
                    0-0 \\
                    ad - bc
                  \end{bmatrix}                   
= 
 \begin{bmatrix}
                    0  \\
                    0  \\
                    ad - bc 
                  \end{bmatrix}                                                   
  \end{align*}
  
\section{Point of Intersection $\in \mathbb{P}^2$}  
\subsection{$x-y + w = 0, w = 0$}
  \begin{align*}
    P &= \begin{bmatrix}
           1 \\
           -1 \\
           1
         \end{bmatrix} \times 
          \begin{bmatrix}
                    0 \\
                    0 \\
                    1
                  \end{bmatrix}  
= 
 \begin{bmatrix}
                    -1 - 0 \\
                    0-1 \\
                    0 - 0
                  \end{bmatrix}                   
= 
 \begin{bmatrix}
                    -1  \\
                    -1 \\
                    0 
                  \end{bmatrix}                                                   
  \end{align*}
\subsection{$3x-w=0 , 4y - w = 0$}
  \begin{align*}
    P &= \begin{bmatrix}
           3 \\
           0 \\
           -1
         \end{bmatrix} \times 
          \begin{bmatrix}
                    0 \\
                    4 \\
                    -1
                  \end{bmatrix}  
= 
 \begin{bmatrix}
                    0+4 \\
                    0+3 \\
                    12-0
                  \end{bmatrix}                   
= 
 \begin{bmatrix}
                    4 \\
                    3 \\
                    12 
                  \end{bmatrix}                                                   
  \end{align*}
\subsection{$x-y+5w=0, x-y+2w$}
  \begin{align*}
    P &= \begin{bmatrix}
           1 \\
           -1 \\
           5
         \end{bmatrix} \times 
          \begin{bmatrix}
                    1 \\
                    -1 \\
                    2
                  \end{bmatrix}  
= 
 \begin{bmatrix}
                    -2+5 \\
                    5-2\\
                    -1+1
                  \end{bmatrix}                   
= 
 \begin{bmatrix}
                    1 \\
                    1 \\
                    0 
                  \end{bmatrix}                                                   
  \end{align*}
  
\section{Find $\lambda$ such that three lines intersect}  

Write the lines in a matrix system of equations:

\begin{align*}
\begin{bmatrix}
0 		& 0 		& 1\\
1 		& \lambda 	& \lambda\\
\lambda & 1 		& \lambda
\end{bmatrix} 
\begin{bmatrix}
x\\
y\\
w
\end{bmatrix} = 
\begin{bmatrix}
0\\
0\\
0
\end{bmatrix}
\end{align*}

Given that $w = 0$, we reduce the system of equations to the following:      
\begin{align*}
\begin{bmatrix}
1 		& \lambda 		\\
\lambda & 1 		
\end{bmatrix} 
\begin{bmatrix}
x\\
y
\end{bmatrix} = 
\begin{bmatrix}
0\\
0
\end{bmatrix}
\end{align*}

Set the determinant equal to zero:
\begin{align*}
1-\lambda &= 0 \\
(1-\lambda)(1 + \lambda ) &= 0 \\
\lambda &= -1, 1
\end{align*}

We choose $\lambda = -1$ since in the other case, we do not have three distinct lines. Using this, we compute the point of intersection as follows:

\begin{align*}
\begin{bmatrix}
0 \\
0 \\ 
0
\end{bmatrix} \times
\begin{bmatrix}
1 \\
-1 \\ 
-1
\end{bmatrix} = 
\begin{bmatrix}
0 +1\\
1+0\\ 
0+0
\end{bmatrix} =
\begin{bmatrix}
1\\
1\\ 
0
\end{bmatrix} 
\end{align*}

\section{Find projective transformation $A$}  

We wish to preserve:
\begin{align*}
P_1 &= (1, 0, 0)\\
P_2 &= (0, 1, 0)\\
O &= (0, 0, 1)
\end{align*}

We wish to map: $P_3 = (1, 1, 1) \to {P_3}' = (3, 2, 1)$

\begin{align*}
({P_1}', {P_2}', {P_3}', O) &= M ({P_1}, {P_2}, {P_3}, O) \\ \\
\begin{bmatrix}
\lambda_1 	& 0 		& 3\lambda_3 	& 0 \\
0 			& \lambda_2 & 2\lambda_3 	& 0 \\
0		 	& 0 		& \lambda_3 	& \lambda_4 
\end{bmatrix}  &=
M \begin{bmatrix}
1 			& 0 	& 1 	& 0 \\
0 			& 1 	& 1 	& 0 \\
0		 	& 0 	& 1 	& 1 
\end{bmatrix} 
\end{align*}

Simplify the matrix by dividing by $\lambda_4$ on both sides:
\begin{align*}
\begin{bmatrix}
a 	& 0 & 3c 	& 0 \\
0 	& b & 2c 	& 0 \\
0	& 0 & c 	& 1
\end{bmatrix}  &=
M' \begin{bmatrix}
1 			& 0 	& 1 	& 0 \\
0 			& 1 	& 1 	& 0 \\
0		 	& 0 	& 1 	& 1 
\end{bmatrix} 
\end{align*}
Compute the matrix product on the RHS of the above equation, and solve for $M'$ as follows:
\begin{align*}
M  &=
 \begin{bmatrix}
a 			& 0 	& 0 	\\
0 			& b 	& 0 	\\
0		 	& 0 	& 1 	
\end{bmatrix}  =
 \begin{bmatrix}
3 			& 0 	& 0 	\\
0 			& 2 	& 0 	\\
0		 	& 0 	& 1 	
\end{bmatrix}  \text{ $\because c = 1$}
\end{align*}

\section{Find projective transformation $P$}  

\begin{align*}
W' \sim P
 \begin{bmatrix}
0 		\\
0 		\\
1		
\end{bmatrix} \implies 
\alpha W' = P
 \begin{bmatrix}
0 		\\
0 		\\
1		
\end{bmatrix}   \\
X' \sim P
 \begin{bmatrix}
0 		\\
1 		\\
0		
\end{bmatrix} \implies 
\beta X' = P
 \begin{bmatrix}
0 		\\
1 		\\
0		
\end{bmatrix}  \\
Y' \sim P
 \begin{bmatrix}
1 		\\
0 		\\
0		
\end{bmatrix} \implies 
\gamma Y' = P
 \begin{bmatrix}
1 		\\
0 		\\
0		
\end{bmatrix}   \\
Z' \sim P
 \begin{bmatrix}
1 		\\
1 		\\
1		
\end{bmatrix} \implies 
\delta Z' = P
 \begin{bmatrix}
1 		\\
1 		\\
1		
\end{bmatrix} 
\end{align*}

Combining the above equations:
\begin{align*}
\delta Z' &= \alpha W' + \beta X' + \gamma Y' \\ \\
\delta  \begin{bmatrix}
1 		\\
1 		\\
1		
\end{bmatrix}  &= \alpha  \begin{bmatrix}
3		\\
0 		\\
1		
\end{bmatrix} +
\beta \begin{bmatrix}
0 		\\
3 		\\
1		
\end{bmatrix} + \gamma  \begin{bmatrix}
-3 		\\
0 		\\
1		
\end{bmatrix} 
\end{align*} 
We can infer that $\delta = 1$, and simplify the system of equations to:
\begin{align*}
\begin{bmatrix}
\alpha & \beta & \gamma
\end{bmatrix}  &= \left(\begin{bmatrix}
3 & 0 & -3 		\\
0 & 3  & 0		\\
1 & 1  & 1	
\end{bmatrix} \right) ^{-1}
\begin{bmatrix}
1		\\
1 		\\
1		
\end{bmatrix}
\end{align*} 

Solving for the inverse of the coefficient matrix, we obtain:
\begin{align*}
\begin{bmatrix}
\frac{1}{6} & -\frac{1}{6} & \frac{1}{2}	\\
0 & \frac{1}{3} & 0		\\
-\frac{1}{6}  & -\frac{1}{6}  & \frac{1}{2}	
\end{bmatrix} \implies 
\begin{bmatrix}
\alpha \\
\beta \\
\gamma
\end{bmatrix} = 
\begin{bmatrix}
\frac{1}{2} \\
\frac{1}{3} \\
\frac{1}{6}
\end{bmatrix} 
\end{align*}

From this, we obtain the transformation $P$ by multiplying $\alpha, \beta$ and $\gamma$ into the above equation:
\begin{align*}
\begin{bmatrix}
\frac{3}{2} & 0 & -\frac{1}{2}	\\
0 & 1 & 0		\\
\frac{1}{2}  & \frac{1}{3}  & \frac{1}{6}	
\end{bmatrix}
\end{align*}

\section{Determine the projective transformation $A$}

Following the same method as the previous question, we set-up the system of equations as follows:
\begin{align*}
\delta  \begin{bmatrix}
1 		\\
1 		\\
1		
\end{bmatrix}  &= \alpha  \begin{bmatrix}
-a		\\
0 		\\
1		
\end{bmatrix} +
\beta \begin{bmatrix}
0 		\\
b 		\\
1		
\end{bmatrix} + \gamma  \begin{bmatrix}
0 		\\
0 		\\
1		
\end{bmatrix} 
\end{align*} 
We can infer that $\delta = 1$, and simplify the system of equations to:
\begin{align*}
\begin{bmatrix}
\alpha & \beta & \gamma
\end{bmatrix}  &= \left(\begin{bmatrix}
-a & 0 & 0 		\\
0 & b  & 0		\\
1 & 1  & 1	
\end{bmatrix} \right) ^{-1}
\begin{bmatrix}
1		\\
1 		\\
1		
\end{bmatrix}
\end{align*} 

Solving for the inverse of the coefficient matrix, we obtain:
\begin{align*}
\begin{bmatrix}
-\frac{1}{a} & 0 & 0	\\
0 & \frac{1}{b} & 0		\\
\frac{1}{a}  & -\frac{1}{b}  & 1	
\end{bmatrix} \implies 
\begin{bmatrix}
\alpha \\
\beta \\
\gamma
\end{bmatrix} = 
\begin{bmatrix}
-\frac{1}{a} \\
\frac{1}{b} \\
\frac{1}{a} - \frac{1}{b} + 1
\end{bmatrix} 
\end{align*}

In the above equation, we know $\gamma = 1$ because the origin is preserved by the transformation. From this result, we obtain the transformation $A^{-1}$ by multiplying $\alpha, \beta$ and $\gamma$ into the above equation:
\begin{align*}
A^{-1} =
\begin{bmatrix}
1 & 0 & 0	\\
0 & 1 & 0		\\
-\frac{1}{a}  & \frac{1}{b}  & 1
\end{bmatrix} \implies 
A = \begin{bmatrix}
1 & 0 & 0	\\
0 & 1 & 0		\\
\frac{1}{a}  & -\frac{1}{b}  & 1
\end{bmatrix}
\end{align*}

\end{document}

